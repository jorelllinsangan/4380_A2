\documentclass{report}

%********************************************
%Packages
%********************************************


\usepackage{graphicx}
\usepackage{amsmath}
\usepackage{algpseudocode}
\usepackage{algorithmicx}
\usepackage{enumitem}
\usepackage{fancyhdr}
\usepackage{indentfirst}
\usepackage{soul}
\usepackage{mathtools}
\usepackage{subfigure}
\usepackage[makeroom]{cancel}
\usepackage{tikz}
\usepackage{qtree}
\usepackage[letterpaper, margin=1in]{geometry}

%********************************
%Package Libraries
%********************************
\usetikzlibrary{matrix}
\usetikzlibrary{arrows}
%********************************
%New Commands
%********************************
\newcommand{\colvec}[2][.8]{%
  \scalebox{#1}{%
    \renewcommand{\arraystretch}{.8}%
    $\begin{bmatrix}#2\end{bmatrix}$%
  }
}

\newcommand{\dc}{$D \text{ \& } C  $}
\DeclarePairedDelimiter{\ceil}{\lceil}{\rceil}
\DeclarePairedDelimiter{\floor}{\lfloor}{\rfloor}

%********************************
%Document Start
%********************************

\begin{document}
	\rhead{\today}
	\title{Machine Learning}
	\author{Jorell Linsangan - 767816 - linsangj@myumanitoba.ca}
	\maketitle
	\newpage
%********************************
%Body
%********************************

	
	\rhead{\today}
	\pagestyle{fancyplain}

	\begin{enumerate}
		\item Bitmap indexes
		\begin{enumerate}
			\item Create a bitmap for each value of Boolean attribute ``evenNum''. \\ 
			\begin{center}
				\begin{tabular} {|c|c|}
					\hline
					$BV(evenNum = Y)$ & $BV(evenNum = N)$ \\ \hline
					1 & 0 \\ \hline
					0 & 1 \\ \hline
					0 & 1 \\ \hline
					0 & 1 \\ \hline
					0 & 1 \\ \hline
					0 & 1 \\ \hline
					0 & 1 \\ \hline
					0 & 1 \\ \hline
					0 & 1 \\ \hline
					0 & 1 \\ \hline
				\end{tabular}	
			\end{center}
			\item Create a bitmap vector for each numeric attribute ``numDigits'' \\
				\begin{center}
				\begin{tabular} {|c|c|}
					\hline
					$BV(numDigits = 1)$ & $BV(numDigits = 2)$ \\ \hline
					1 & 0 \\ \hline
					1 & 0 \\ \hline
					1 & 0 \\ \hline
					1 & 0 \\ \hline
					0 & 1 \\ \hline
					0 & 1 \\ \hline
					0 & 1 \\ \hline
					0 & 1 \\ \hline
					0 & 1 \\ \hline
					0 & 1 \\ \hline
				\end{tabular}	
			\end{center}
			\item Create a bitmap vector for each Boolean attribute ``endsWith3'' \\
				\begin{center}
				\begin{tabular} {|c|c|}
					\hline
					$BV(endsWith3 = Y)$ & $BV(endsWith3 = N)$ \\ \hline
					0 & 1 \\ \hline
					1 & 0 \\ \hline
					0 & 1 \\ \hline
					0 & 1 \\ \hline
					0 & 1 \\ \hline
					1 & 0 \\ \hline
					0 & 1 \\ \hline
					0 & 1 \\ \hline
					1 & 0 \\ \hline
					0 & 1 \\ \hline
				\end{tabular}	
			\end{center}
			\item Find the rIDs of all the records with 2-digit values that end with 3.
			\begin{center}
				\begin{align*}
					BV(numDigits=2) &= 0000111111 \\
					BV(endWith3 =Y ) &= 0100010010 \\
					BV(Result) &= 0000010010
				\end{align*}
			\end{center}
			Which means that the rids of records with 2-digit values that end with 3 are $6$ \& $9$.
		\end{enumerate}
		\newpage
		\item I/O Cost
		\begin{enumerate}
			\item 2-way external mergesort.
			\begin{align*}
				Cost &= 2(N)(1 + \lceil\log_2(N)\rceil) \\
				&= 2(200)(1 + \lceil\log_2(200)\rceil)\\
				&= 400(1 +\lceil\log_2(200)\rceil) \\
				&= 400 + 3200 \\
				\text{Total Cost} &= 3600 \text{ I/Os}
			\end{align*}
			\item General multi-way external mergesort with B = 15
			\begin{align*}
				Cost &= 2(N)(1 + \lceil\log_{B-1}(\lceil\dfrac{N}{B}\rceil)\rceil) \\
				&= 2(200)(1 + \lceil\log_{14}(\lceil\dfrac{200}{15}\rceil)\rceil) \\
				&= 400(2) \\
				\text{Total Cost} &= 800 \text{ I/Os}
			\end{align*}
		\end{enumerate}
		\item Something...Not sure what to name this question. Query Processing?
		If there are $2000$ tuples that are sorted in ascending order and a page can hold 10 student tuples. $P_r = 200$.
		\begin{enumerate}
			\item Binary Search. I am assuming that sID is the primary key, therefor it is unique.
			\begin{align*}
				\text{Cost to locate tuple} &= \lceil \log_2(200) \rceil \\
				&= 8 \text{ I/Os}
			\end{align*}

			\item $3$ I/Os
			\item $4$ I/Os
		\end{enumerate}
		\item 
		\begin{enumerate}
			\item Root to leaf traversal is $3$. There are 200 pages and there are 2000 tuples where 2\% of those are `G. Raymond'. So $3 + (.02)200 = 7$ I/Os for clustered index.
			\item For non-clustered, it is $3 + (0.02)2000 = 43$ I/Os. *
		\end{enumerate}
		\item 
		\begin{enumerate}
			\item We create a tuple of only the wanted attribute/s which is `sName'. Let this tuple be tuple T. Since size(sName) == size(sID), we can assume that we can fit 20 student tuples in a page now. So $P_t = 2000/20 = 100$ and $P_E = 2000/10 = 200$. The formula for cost is:
			\begin{align*}
				&P_E + P_T + ( 2(P_T) \lceil(\log_{B-1} \lceil\dfrac{P_T}{B}\rceil)\rceil - 1) + P_T \\
				&200 + 100 + ( 2(100) \lceil(\log_{20} \lceil\dfrac{100}{21}\rceil)\rceil -1 ) + 100 = 400 \\
			\end{align*}
			\item $P_E + 2 P_T$ is the formula to get the number of I/Os for hash based projection. $P_E = 200$ and depending if I should assume that size(sName) == size(sID). $P_T$ will have different values. $P_T = 100$ if I make the assumption or $P_T = 200$ if no assumption. \\

			Answer with assumption: $200+ (2)(100) = 400$ \\
			Answer without assumption $200 + (2)(200) = 600$
		\end{enumerate}
		\newpage
		\item $P_S = 200$, $P_E = 800$
		\begin{enumerate}
			\item Inner: Enrolled, Outer: Student. Cost: $P_S + N_S * P_E = 200 + 2000(800) = 1600200$.
			\item Inner: Student, Outer: Enrolled. Cost: $P_E + N_R * P_S = 800 + 8000(200) = 1600800$.
			\item $P_E + P_E(P_S) = 800 + 800(200) = 160800$
			\item ``Block'' nested-loop join. \\
			With $E$ as outer. $B-2 = 19$ pages of $E$ per chunk. \\
			Cost of scanning E: $P_E = 800$ pages. \\
			Number of chunks = $\lceil \dfrac{800}{19} \rceil = 43 $ chunks. \\
			Per chunk of E, we scan S: $\lceil \dfrac{P_E}{B-2} \rceil * P_S = 43(200)$ pages.  \\
			Total cost = $800 + 8600 = 9400$ pages.
			\item Sort-merge join. \\
			Cost for sorting enrolled is $2(800) * (1 + \lceil\log_{20} \lceil \dfrac{800}{21}\rceil\rceil) = 4800$ \\
			The cost for the merging phase is $P_E + P_S = 800 + 200 = 1000$. \\
			Cost for Sort-merge join is $4800 + 1000 = 5800$.
		\end{enumerate}

		\item Schedules
		\begin{enumerate}
			\item Schedule 1:
			\begin{enumerate}
				\item Yes
				\item Yes \\
				\begin{tabular}{c c} \\
				T1 & T2 \\
				\hline 
				R(A) & \\
				W(A) & \\ 
				R(B) & \\ 
				W(B) & \\
 					 & R(A) \\
 					 & W(A)
				\end{tabular}
				\item Yes \\
				\begin{tabular}{c c} \\
				T1 & T2 \\
				\hline 
				R(A) & \\
				W(A) & \\ 
				R(B) & \\ 
				W(B) & \\
 					 & R(A) \\
 					 & W(A)
				\end{tabular}
			\end{enumerate}
			\item Schedule 2:
			\begin{enumerate}
				\item No
				\item Yes\\
				\begin{tabular}{c c} \\
				T3 & T4 \\
				\hline 
				& R(C) \\
				& W(C) \\
				& R(D) \\ 
				& W(D)\\ 
				R(C) & \\ 
				W(C) & \\
				\end{tabular}
				\item Yes \\
				\begin{tabular}{c c} \\
				T3 & T4 \\
				\hline 
				& R(C) \\
				& W(C) \\
				& R(D) \\ 
				& W(D)\\ 
				R(C) & \\ 
				W(C) & \\
				\end{tabular}
			\end{enumerate}
			\item Schedule 3:
			\begin{enumerate}
				\item No
				\item No
				\item Yes \\
				\begin{tabular}{c c} \\
				T5 & T6 \\
				\hline 
				R(E)& \\
				W(E)& \\ 
				& W(E)) \\
				& W(E)\\
				\end{tabular}
			\end{enumerate}
			\item Schedule 4:
			\begin{enumerate}
				\item No
				\item No
				\item No
			\end{enumerate}
		\end{enumerate}
	\end{enumerate}
\end{document}